\documentclass{article}


\usepackage{PRIMEarxiv}

\usepackage[utf8]{inputenc} % allow utf-8 input
\usepackage[T1]{fontenc}    % use 8-bit T1 fonts
\usepackage{hyperref}       % hyperlinks
\usepackage{url}            % simple URL typesetting
\usepackage{booktabs}       % professional-quality tables
\usepackage{amsfonts}       % blackboard math symbols
\usepackage{nicefrac}       % compact symbols for 1/2, etc.
\usepackage{microtype}      % microtypography
\usepackage{lipsum}
\usepackage{fancyhdr}       % header
\usepackage{graphicx}       % graphics
\graphicspath{{media/}}     % organize your images and other figures under media/ folder

%Header
\pagestyle{fancy}
\thispagestyle{empty}
\rhead{ \textit{ }} 

% Update your Headers here
\fancyhead[LO]{Running Title for Header}
% \fancyhead[RE]{Firstauthor and Secondauthor} % Firstauthor et al. if more than 2 - must use \documentclass[twoside]{article}



  
%% Title
\title{Comparative Analysis of Diffusion Models for Image Generation
%%%% Cite as
%%%% Update your official citation here when published 
%\thanks{\textit{\underline{Citation}}: 
%\textbf{V. Sharan. Comparative Analysis of Diffusion Models for Image Generation.}} 
}

\author{
  Vaibhav Sharan\\
  \texttt{vsharan1@asu.edu} \\ 
  \and
  Krishnaprasad Palamattam Aji\\
  \texttt{kpalamat@asu.edu}\\
  \and
  Unnikrishnan Madhavan\\
  \texttt{umadhava@asu.edu}\\
  \and
  Ansh Sharma\\
  \texttt{ansh@asu.edu}
}


\begin{document}
\maketitle

\section{Introduction}
The project aims at implementing and comparing different diffusion models that are used for 
image generation. We will evaluate the performance of these different models in generating
such high-quality images from textual prompts. The well known models like FLUX, Stable
Diffusion will be studied in detail and compared with each other on the basis of
their strengths, weaknesses and applicability in different scenarios. This study aims to shed light
on the effectiveness of text to image diffusion models.


\section{Background and Motivation}
Artificial Intelligence has witnessed a paradigm shift in techniques for image generation over the past few years. 
Generative Adversarial Networks(GANs) have been the go to approach for synthetic image creation for a long time. 
Diffusion models recently emerged as a powerful alternative, especially in the domain of text-to-image conversion \cite{dhariwal2021}. 

Diffusion models were introduced in 2015 by Sohl-Dickstein et al.\cite{sohl2015} and have gained popularity 
due to their ability to generate diverse images with high quality and remarkable fidelity. GANs rely on a 
generator-discriminator (adversarial) architecture to generate images. Unlike GANs, diffusion models use a 
gradual denoising process to generate images which has shown remarkable stability during training and better control
over the generation process.

Diffusion models gained popularity due to their exceptional performance in generating images from text. Models like DALL-E 2,
Stable Diffusion, FLUX.1 and Midjourney have captured public imagination with their ability to form real like images 
from textual descriptions. Our project "Comparative Analysis of Diffusion Models for Image Generation", is motivated by the 
need to understand the strengths and weaknesses of different diffusion model architectures. These models are evolving rapidly, and a 
comprehensive comparison is crucial and beneficial to both researchers and practitioners in this field.





\section{Related Work}
Literature Review goes here

\section{Progress}


\section{Execution Plan}

\section{Workload Distribution}

\section{Headings: first level}
\label{sec:headings}

\lipsum[4] See Section \ref{sec:headings}.

\subsection{Headings: second level}
\lipsum[5]
\begin{equation}
\xi _{ij}(t)=P(x_{t}=i,x_{t+1}=j|y,v,w;\theta)= {\frac {\alpha _{i}(t)a^{w_t}_{ij}\beta _{j}(t+1)b^{v_{t+1}}_{j}(y_{t+1})}{\sum _{i=1}^{N} \sum _{j=1}^{N} \alpha _{i}(t)a^{w_t}_{ij}\beta _{j}(t+1)b^{v_{t+1}}_{j}(y_{t+1})}}
\end{equation}

\subsubsection{Headings: third level}
\lipsum[6]

\paragraph{Paragraph}
\lipsum[7]

\section{Examples of citations, figures, tables, references}
\label{sec:others}
\lipsum[8] \cite{dhariwal2021,kour2014fast} and see \cite{hadash2018estimate}.

The documentation for \verb+natbib+ may be found at
\begin{center}
  \url{http://mirrors.ctan.org/macros/latex/contrib/natbib/natnotes.pdf}
\end{center}
Of note is the command \verb+\citet+, which produces citations
appropriate for use in inline text.  For example,
\begin{verbatim}
   \citet{hasselmo} investigated\dots
\end{verbatim}
produces
\begin{quote}
  Hasselmo, et al.\ (1995) investigated\dots
\end{quote}

\begin{center}
  \url{https://www.ctan.org/pkg/booktabs}
\end{center}


\subsection{Figures}
\lipsum[10] 
See Figure \ref{fig:fig1}. Here is how you add footnotes. \footnote{Sample of the first footnote.}
\lipsum[11] 

\begin{figure}
  \centering
  \fbox{\rule[-.5cm]{4cm}{4cm} \rule[-.5cm]{4cm}{0cm}}
  \caption{Sample figure caption.}
  \label{fig:fig1}
\end{figure}

\subsection{Tables}
\lipsum[12]
See awesome Table~\ref{tab:table}.

\begin{table}
 \caption{Sample table title}
  \centering
  \begin{tabular}{lll}
    \toprule
    \multicolumn{2}{c}{Part}                   \\
    \cmidrule(r){1-2}
    Name     & Description     & Size ($\mu$m) \\
    \midrule
    Dendrite & Input terminal  & $\sim$100     \\
    Axon     & Output terminal & $\sim$10      \\
    Soma     & Cell body       & up to $10^6$  \\
    \bottomrule
  \end{tabular}
  \label{tab:table}
\end{table}

\subsection{Lists}
\begin{itemize}
\item Lorem ipsum dolor sit amet
\item consectetur adipiscing elit. 
\item Aliquam dignissim blandit est, in dictum tortor gravida eget. In ac rutrum magna.
\end{itemize}


\section{Conclusion}
Your conclusion here

\section*{Acknowledgments}
This was was supported in part by......

%Bibliography
\bibliographystyle{unsrt}  
\bibliography{references}  


\end{document}
